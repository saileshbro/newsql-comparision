\chapter{Wide-Column Databases using Apache Cassandra}

Apache Cassandra represents a different approach to distributed data storage, built for high availability and horizontal scaling \parencite{cassandra_architecture}. This wide-column store organizes data in tables with rows and dynamic columns—each row can maintain different column sets \parencite{cassandra_documentation}. Cassandra's architecture excels at write-heavy workloads and provides eventual consistency, making it ideal for applications that prioritize availability and horizontal scaling over immediate consistency \parencite{cassandra_architecture}.

\section{Database Schema}

\subsection{Keyspace and Table Design}
Our database schema works well with typical query patterns in an attendance system. The complete schema definition is shown below:

\begin{minted}[breaklines=true,frame=lines,linenos]{sql}
-- Create keyspace
CREATE KEYSPACE IF NOT EXISTS university
WITH replication = {
    'class': 'SimpleStrategy',
    'replication_factor': 1
};

-- Use the keyspace
USE university;

-- Create attendance table
-- Partition key: student_id (groups attendance records by student)
-- Clustering keys: course_code, date (orders records within partition)
CREATE TABLE IF NOT EXISTS attendance (
    student_id text,
    course_code text,
    date date,
    present boolean,
    PRIMARY KEY (student_id, course_code, date)
) WITH CLUSTERING ORDER BY (course_code ASC, date DESC);

-- Create index for querying by course_code
CREATE INDEX IF NOT EXISTS idx_attendance_course
ON attendance (course_code);

-- Create index for querying by date
CREATE INDEX IF NOT EXISTS idx_attendance_date
ON attendance (date);

-- Display table schema
DESCRIBE TABLE attendance;
\end{minted}

\subsection{Schema Design Rationale}

\subsubsection{Primary Key Design}
The primary key consists of:
\begin{itemize}
    \item \textbf{Partition Key:} \texttt{student\_id} - Groups all attendance records for a student together
    \item \textbf{Clustering Keys:} \texttt{course\_code}, \texttt{date} - Orders records within each partition
\end{itemize}

This design supports fast queries for:
\begin{itemize}
    \item All attendance records for a specific student
    \item Attendance records for a student in a specific course
    \item Attendance records for a student on specific dates
\end{itemize}

\subsubsection{Secondary Indexes}
Two secondary indexes support additional query patterns:
\begin{itemize}
    \item \texttt{idx\_attendance\_course} - Supports queries by \texttt{course\_code}
    \item \texttt{idx\_attendance\_date} - Supports queries by date range
\end{itemize}

\section{Data Model}

\subsection{Sample Data Structure}
The system uses an attendance dataset covering multiple students across different departments and courses. Below is a representative sample of the data structure (complete dataset available in Appendix B):

\begin{minted}[breaklines=true,frame=lines,linenos]{typescript}
export const attendanceData = [
  // Student CS001 attendance records
  { student_id: 'CS001', course_code: 'CS101', date: '2024-01-15', present: true },
  { student_id: 'CS001', course_code: 'CS101', date: '2024-01-16', present: false },
  { student_id: 'CS001', course_code: 'CS102', date: '2024-01-15', present: true },

  // Student CS002 attendance records
  { student_id: 'CS002', course_code: 'CS101', date: '2024-01-15', present: false },
  { student_id: 'CS002', course_code: 'CS103', date: '2024-01-15', present: true },

      // Students from other departments: IT001, EE001, ME001
  // Complete dataset available in Appendix B (Section B.2)
  // Total: 22 attendance records across 5 students and 8 courses
];
\end{minted}

\section{Database Operations}

\subsection{Database Setup and Connection}

\subsubsection{Connection Configuration}
\begin{minted}[breaklines=true,frame=lines,linenos]{typescript}
import { Client } from 'cassandra-driver';

export async function connectToCassandra(): Promise<Client> {
  const client = new Client({
    contactPoints: ['127.0.0.1'],
    localDataCenter: 'datacenter1',
    keyspace: 'university'
  });

  try {
    await client.connect();
    console.log('[SUCCESS] Connected to Cassandra');
    return client;
  } catch (error) {
    console.error('[ERROR] Error connecting to Cassandra:', error);
    throw error;
  }
}
\end{minted}

\subsubsection{Database Initialization}
The setup process includes creating the keyspace, table, and indexes as shown in Figure~\ref{fig:task2-create-database}.

\subsection{Data Insertion Operations}

\subsubsection{Bulk Data Insertion}
\begin{minted}[breaklines=true,frame=lines,linenos]{typescript}
async function insertAttendanceData() {
  let client: any;
  try {
    client = await connectToCassandra();

    console.log('[INFO] Inserting attendance data...');

    // Prepare the insert statement
    const insertQuery = `
      INSERT INTO attendance (student_id, course_code, date, present)
      VALUES (?, ?, ?, ?)
    `;

    // Insert each record
    for (const record of attendanceData) {
      await client.execute(insertQuery, [
        record.student_id,
        record.course_code,
        record.date,
        record.present
      ]);
      console.log(`[SUCCESS] Inserted: ${record.student_id} - ${record.course_code} - ${record.date} - ${record.present ? 'Present' : 'Absent'}`);
    }

    console.log(`[SUCCESS] Successfully inserted ${attendanceData.length} attendance records!`);

    // Display total count
    const countResult = await client.execute('SELECT COUNT(*) FROM attendance');
    console.log(`[INFO] Total attendance records in database: ${countResult.rows[0].count}`);

  } catch (error) {
    console.error('[ERROR] Error inserting data:', error);
  } finally {
    if (client) {
      await disconnectFromCassandra(client);
    }
  }
}
\end{minted}

The data insertion process is demonstrated in Figure~\ref{fig:task2-insert-data}.

\section{Query Operations}

\subsection{Basic Queries}

\subsubsection{Count Query}
\begin{minted}[breaklines=true,frame=lines,linenos]{sql}
SELECT COUNT(*) FROM attendance;
\end{minted}

The count query returns the total number of attendance records in the database:

\begin{table}[H]
\centering
\begin{tabular}{|c|}
\hline
\textbf{COUNT(*)} \\
\hline
\texttt{22} \\
\hline
\end{tabular}
\caption{Count Query Result - Total Attendance Records}
\end{table}

This confirms that all 22 attendance records have been successfully inserted into the database. The complete execution screenshot is available in Appendix B (Figure~\ref{fig:task2-count-query}).

\subsubsection{Query by Student ID}
\begin{minted}[breaklines=true,frame=lines,linenos]{sql}
SELECT * FROM attendance WHERE student_id = 'CS001';
\end{minted}

This query retrieves all attendance records for a specific student, using the partition key design. The \texttt{SELECT} query returns:

\begin{table}[H]
  \centering
  \footnotesize
  \begin{tabular}{|c|c|c|c|}
    \hline
    \textbf{student\_id} & \textbf{course\_code} & \textbf{date} & \textbf{present} \\
    \hline
    \texttt{CS001}                & \texttt{CS101}                 & \texttt{2024-01-17}    & \texttt{true}             \\
    \texttt{CS001}                & \texttt{CS101}                 & \texttt{2024-01-16}    & \texttt{false}            \\
    \texttt{CS001}                & \texttt{CS101}                 & \texttt{2024-01-15}    & \texttt{true}             \\
    \texttt{CS001}                & \texttt{CS102}                 & \texttt{2024-01-16}    & \texttt{true}             \\
    \texttt{CS001}                & \texttt{CS102}                 & \texttt{2024-01-15}    & \texttt{true}             \\
    \hline
  \end{tabular}
  \caption{Query results for student CS001 (5 records returned)}
\end{table}

Note the clustering order: records are ordered by course\_code ASC, then date DESC within each course. The complete execution screenshot is available in Appendix B (Figure~\ref{fig:task2-select-student}).

\subsubsection{Query by Student ID and Course}
\begin{minted}[breaklines=true,frame=lines,linenos]{sql}
SELECT * FROM attendance WHERE student_id = 'CS001' AND course_code = 'CS101';
\end{minted}

This query uses both the partition key and clustering key according to Cassandra's query optimization design. The \texttt{SELECT} result set:

\begin{table}[H]
  \centering
  \footnotesize
  \begin{tabular}{|c|c|c|c|}
    \hline
    \textbf{student\_id} & \textbf{course\_code} & \textbf{date} & \textbf{present} \\
    \hline
    \texttt{CS001}                & \texttt{CS101}                 & \texttt{2024-01-17}    & \texttt{true}             \\
    \texttt{CS001}                & \texttt{CS101}                 & \texttt{2024-01-16}    & \texttt{false}            \\
    \texttt{CS001}                & \texttt{CS101}                 & \texttt{2024-01-15}    & \texttt{true}             \\
    \hline
  \end{tabular}
  \caption{Query results for student CS001 in course CS101 (3 records returned)}
\end{table}

This shows how compound key queries work, retrieving only the specific student-course combination. The \texttt{WHERE} clause with \texttt{AND} conditions demonstrates Cassandra's compound key functionality. The execution screenshot is available in Appendix B (Figure~\ref{fig:task2-select-student-course}).

\subsection{Extended Query Operations}

\subsubsection{Date Range Queries}
Using secondary indexes, the system supports queries by date ranges. For example, querying attendance records for a specific date using \texttt{SELECT}:

\begin{minted}[breaklines=true,frame=lines,linenos]{sql}
SELECT * FROM attendance WHERE date = '2024-01-15' ALLOW FILTERING;
\end{minted}

\begin{table}[H]
  \centering
  \footnotesize
  \begin{tabular}{|c|c|c|c|}
    \hline
    \textbf{student\_id} & \textbf{course\_code} & \textbf{date} & \textbf{present} \\
    \hline
    \texttt{CS001}                & \texttt{CS101}                 & \texttt{2024-01-15}    & \texttt{true}             \\
    \texttt{CS001}                & \texttt{CS102}                 & \texttt{2024-01-15}    & \texttt{true}             \\
    \texttt{CS002}                & \texttt{CS101}                 & \texttt{2024-01-15}    & \texttt{false}            \\
    \texttt{CS002}                & \texttt{CS103}                 & \texttt{2024-01-15}    & \texttt{true}             \\
    \texttt{IT001}                & \texttt{IT201}                 & \texttt{2024-01-15}    & \texttt{true}             \\
    \texttt{IT001}                & \texttt{IT202}                 & \texttt{2024-01-15}    & \texttt{false}            \\
    \texttt{EE001}                & \texttt{EE301}                 & \texttt{2024-01-15}    & \texttt{true}             \\
    \texttt{EE001}                & \texttt{EE302}                 & \texttt{2024-01-15}    & \texttt{true}             \\
    \texttt{ME001}                & \texttt{ME401}                 & \texttt{2024-01-15}    & \texttt{false}            \\
    \texttt{ME001}                & \texttt{ME402}                 & \texttt{2024-01-15}    & \texttt{true}             \\
    \hline
  \end{tabular}
  \caption{All attendance records for January 15, 2024 (10 records)}
\end{table}

This query uses the secondary index on the \texttt{date} field for cross-partition queries. The \texttt{ALLOW FILTERING} clause supports this operation. The execution screenshot is available in Appendix B (Figure~\ref{fig:task2-date-range}).

\subsubsection{Course-Based Queries}
The \texttt{course\_code} index supports fast queries across all students for specific courses:

\begin{minted}[breaklines=true,frame=lines,linenos]{sql}
SELECT * FROM attendance WHERE course_code = 'CS101' ALLOW FILTERING;
\end{minted}

This \texttt{SELECT} query returns all students enrolled in \texttt{CS101}:

\begin{table}[H]
  \centering
  \footnotesize
  \begin{tabular}{|c|c|c|c|}
    \hline
    \textbf{student\_id} & \textbf{course\_code} & \textbf{date} & \textbf{present} \\
    \hline
    \texttt{CS001}                & \texttt{CS101}                 & \texttt{2024-01-17}    & \texttt{true}             \\
    \texttt{CS001}                & \texttt{CS101}                 & \texttt{2024-01-16}    & \texttt{false}            \\
    \texttt{CS001}                & \texttt{CS101}                 & \texttt{2024-01-15}    & \texttt{true}             \\
    \texttt{CS002}                & \texttt{CS101}                 & \texttt{2024-01-17}    & \texttt{true}             \\
    \texttt{CS002}                & \texttt{CS101}                 & \texttt{2024-01-16}    & \texttt{true}             \\
    \texttt{CS002}                & \texttt{CS101}                 & \texttt{2024-01-15}    & \texttt{false}            \\
    \hline
  \end{tabular}
  \caption{Course-based query results for CS101 (6 records)}
\end{table}

The execution screenshot is available in Appendix B (Figure~\ref{fig:task2-where-query}).

\subsubsection{Grouping and Aggregation}
Cassandra supports various grouping operations for analytical queries. For example, counting attendance by student using \texttt{GROUP BY}:

\begin{minted}[breaklines=true,frame=lines,linenos]{sql}
SELECT student_id, COUNT(*) as total_records
FROM attendance
GROUP BY student_id;
\end{minted}

This produces the following aggregated results:

\begin{table}[H]
  \centering
  \begin{tabular}{|c|c|}
    \hline
    \textbf{student\_id} & \textbf{total\_records} \\
    \hline
    \texttt{CS001}                & \texttt{5}                       \\
    \texttt{CS002}                & \texttt{5}                       \\
    \texttt{IT001}                & \texttt{4}                       \\
    \texttt{EE001}                & \texttt{4}                       \\
    \texttt{ME001}                & \texttt{4}                       \\
    \hline
  \end{tabular}
  \caption{Attendance record count by student}
\end{table}

This shows Cassandra's ability to perform grouping operations when the \texttt{GROUP BY} clause matches the partition key. The \texttt{COUNT(*)} aggregation operates following Cassandra's grouping design. The execution screenshot is available in Appendix B (Figure~\ref{fig:task2-student-course-group}).

\section{Performance Analysis}

The Cassandra implementation revealed several key operational characteristics in this attendance tracking workload. Cassandra's architecture targets specific performance patterns that align with distributed data management requirements \parencite{cassandra_architecture}. The complete source code and implementation can be found at: \url{https://github.com/saileshbro/newsql-comparision/blob/main/task-2/src/attendance_data.ts}.

\begin{itemize}
    \item \textbf{Partition Key Efficiency}: Queries using the partition key (student\_id) successfully executed with Cassandra's documented access patterns \parencite{cassandra_documentation}
    \item \textbf{Clustering Key Operations}: Compound queries using both partition and clustering keys operated as intended within Cassandra's data model \parencite{cassandra_architecture}
    \item \textbf{Secondary Index Functionality}: Queries using secondary indexes (course\_code, date) provided cross-partition access capabilities \parencite{cassandra_documentation}
    \item \textbf{Write Operations}: Bulk insert operations successfully handled time-series data insertion following Cassandra's write-optimized design \parencite{cassandra_architecture}
    \item \textbf{Scalability Design}: The partition key design supports Cassandra's documented horizontal scaling architecture \parencite{cassandra_documentation}
\end{itemize}

\section{Use Cases and Applications}

Cassandra's wide-column architecture supports several application patterns \parencite{cassandra_architecture}:

\begin{itemize}
    \item \textbf{Time-Series Data}: The clustering by date supports chronological data access following Cassandra's time-series design \parencite{cassandra_documentation}
    \item \textbf{Event Logging}: High-write throughput with partition-based distribution \parencite{cassandra_architecture}
    \item \textbf{Recommendation Systems}: Storage and retrieval of user-item interactions \parencite{cassandra_documentation}
    \item \textbf{Sensor Data Collection}: Scalable storage for IoT and monitoring applications \parencite{cassandra_architecture}
    \item \textbf{Analytics Platforms}: Support for large-scale data aggregation and analysis \parencite{cassandra_documentation}
\end{itemize}

The attendance tracking system implementation illustrates how Cassandra handles time-series data within its wide-column architecture, showing the database's design principles for distributed data management and availability-focused applications \parencite{cassandra_architecture}.