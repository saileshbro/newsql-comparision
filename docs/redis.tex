\chapter{Key-Value Stores using Redis}

Redis (Remote Dictionary Server) serves as an open-source, in-memory data structure store that functions as a database, cache, and message broker \parencite{redis_paper}. Redis supports various data structures including strings, hashes, lists, sets, sorted sets, and more \parencite{redis_documentation}. Its in-memory architecture delivers low-latency data access patterns \parencite{redis_paper}. Redis excels at applications requiring rapid data access, caching, session management, and real-time analytics \parencite{redis_documentation}.

\section{Basic Key-Value Operations}

\subsection{Simple String Operations}

Redis strings are the most basic Redis data type, representing a sequence of bytes. The following operations show setting and getting simple key-value pairs using \texttt{SET} and \texttt{GET} commands:

\begin{minted}[linenos,breaklines]{text}
# Set simple string values
SET student:1001 "John Doe"
SET student:1002 "Jane Smith"
SET student:1003 "Bob Johnson"

# Get simple values
GET student:1001
GET student:1002
GET student:1003

# Set multiple keys at once
MSET course:CS101 "Introduction to Programming" course:CS102 "Data Structures" course:CS103 "Algorithms"

# Get multiple keys at once
MGET course:CS101 course:CS102 course:CS103
\end{minted}

\begin{table}[H]
  \centering
  \begin{tabular}{|l|l|}
    \hline
    \textbf{Command} & \textbf{Result} \\
    \hline
    SET student:\texttt{1001} "\texttt{John Doe}" & \texttt{OK} \\
    SET student:\texttt{1002} "\texttt{Jane Smith}" & \texttt{OK} \\
    SET student:\texttt{1003} "\texttt{Bob Johnson}" & \texttt{OK} \\
    GET student:\texttt{1001} & "\texttt{John Doe}" \\
    GET student:\texttt{1002} & "\texttt{Jane Smith}" \\
    GET student:\texttt{1003} & "\texttt{Bob Johnson}" \\
    \hline
  \end{tabular}
  \caption{Results from basic string operations. See Appendix Figure~\ref{fig:basic-set-get} for screenshot.}
\end{table}

\subsection{Hash Operations}

Redis hashes are maps between string fields and string values, making them suitable for representing objects like user profiles or student records using \texttt{HSET}, \texttt{HGET}, and \texttt{HGETALL} commands:

\begin{minted}[linenos,breaklines]{text}
# Create hash for student profile
HSET student:profile:1001 name "John Doe" age 20 major "Computer Science" gpa 3.8
HSET student:profile:1002 name "Jane Smith" age 21 major "Information Technology" gpa 3.9
HSET student:profile:1003 name "Bob Johnson" age 19 major "Software Engineering" gpa 3.7

# Get entire hash
HGETALL student:profile:1001
HGETALL student:profile:1002

# Get specific fields from hash
HGET student:profile:1001 name
HMGET student:profile:1002 name major gpa

# Set multiple hash fields
HMSET student:profile:1004 name "Alice Brown" age 22 major "Data Science" gpa 3.95 year "Senior"

# Get all hash keys and values
HKEYS student:profile:1001
HVALS student:profile:1002

# Check if hash field exists
HEXISTS student:profile:1001 name
HEXISTS student:profile:1001 email

# Increment numeric field in hash
HINCRBY student:profile:1001 age 1
\end{minted}

\begin{table}[H]
  \centering
  \begin{tabular}{|l|l|}
    \hline
    \textbf{Operation} & \textbf{Result} \\
    \hline
    HSET student:profile:\texttt{1001} ... & \texttt{4} fields set \\
    HGET student:profile:\texttt{1001} name & "\texttt{John Doe}" \\
    HMGET student:profile:\texttt{1002} name major gpa & ["\texttt{Jane Smith}", "\texttt{Information Technology}", "\texttt{3.9}"] \\
    HEXISTS student:profile:\texttt{1001} name & \texttt{1} (\texttt{true}) \\
    HEXISTS student:profile:\texttt{1001} email & \texttt{0} (\texttt{false}) \\
    HINCRBY student:profile:\texttt{1001} age \texttt{1} & \texttt{21} \\
    \hline
  \end{tabular}
  \caption{Results from hash operations. See Appendix Figure~\ref{fig:hash-operations} for screenshot.}
\end{table}

\section{Session Management with TTL}

One of Redis's key features is the ability to automatically expire keys after a specified time period \parencite{redis_ttl}. This makes it ideal for session management, caching, and temporary data storage \parencite{redis_documentation}.

\subsection{Basic Session Creation}

The following commands show creating user sessions with automatic expiration using \texttt{SET} with \texttt{EX} and \texttt{TTL}:

\begin{minted}[linenos,breaklines]{text}
# Create login sessions with 30 second TTL
SET session:user1001 "John Doe logged in" EX 30
SET session:user1002 "Jane Smith logged in" EX 30
SET session:user1003 "Bob Johnson logged in" EX 30

# Check current sessions
GET session:user1001
GET session:user1002
GET session:user1003

# Check TTL for sessions
TTL session:user1001
TTL session:user1002
TTL session:user1003
\end{minted}

\begin{table}[H]
  \centering
  \begin{tabular}{|l|l|}
    \hline
    \textbf{Command} & \textbf{Result} \\
    \hline
    SET session:user\texttt{1001} "..." EX \texttt{30} & \texttt{OK} \\
    GET session:user\texttt{1001} & "\texttt{John Doe logged in}" \\
    TTL session:user\texttt{1001} & \texttt{11} (seconds remaining) \\
    TTL session:user\texttt{1002} & \texttt{11} (seconds remaining) \\
    \hline
  \end{tabular}
  \caption{Results from basic session management. See Appendix Figure~\ref{fig:session-details} for screenshot.}
\end{table}

\subsection{Detailed Session Data}

For more complex session management, Redis hashes can store detailed session information while maintaining TTL functionality using \texttt{HSET} with \texttt{EXPIRE}:

\begin{minted}[linenos,breaklines]{text}
# Create detailed session data using hashes with TTL
HSET session:detailed:user1001 user_id 1001 username "john_doe" login_time "2024-01-15 10:30:00" ip_address "192.168.1.100"
EXPIRE session:detailed:user1001 45

HSET session:detailed:user1002 user_id 1002 username "jane_smith" login_time "2024-01-15 10:35:00" ip_address "192.168.1.101"
EXPIRE session:detailed:user1002 45

# Check detailed session data
HGETALL session:detailed:user1001
TTL session:detailed:user1001

# Create shopping cart session with TTL
HSET cart:session:user1001 item1 "Laptop" item2 "Mouse" item3 "Keyboard" total 1500
EXPIRE cart:session:user1001 300

# Check cart session
HGETALL cart:session:user1001
TTL cart:session:user1001
\end{minted}

\begin{table}[H]
  \centering
  \begin{tabular}{|l|l|}
    \hline
    \textbf{Field} & \textbf{Value} \\
    \hline
    user\_id & \texttt{1001} \\
    username & \texttt{john\_doe} \\
    login\_time & \texttt{2024-01-15 10:30:00} \\
    ip\_address & \texttt{192.168.1.100} \\
    TTL & \texttt{32} seconds \\
    \hline
  \end{tabular}
  \caption{Detailed session data structure.}
\end{table}

\begin{table}[H]
  \centering
  \begin{tabular}{|l|l|}
    \hline
    \textbf{Cart Item} & \textbf{Value} \\
    \hline
    item\texttt{1} & \texttt{Laptop} \\
    item\texttt{2} & \texttt{Mouse} \\
    item\texttt{3} & \texttt{Keyboard} \\
    total & \texttt{1500} \\
    TTL & \texttt{293} seconds \\
    \hline
  \end{tabular}
  \caption{Shopping cart session data with 5-minute TTL.}
\end{table}

\subsection{Session Expiration}

After the TTL expires, Redis automatically removes the keys. The following shows checking expired sessions using \texttt{GET} and \texttt{TTL}:

\begin{minted}[linenos,breaklines]{text}
# After 30+ seconds, check session expiration
GET session:user1001
GET session:user1002
TTL session:user1001
TTL session:user1002
\end{minted}

\begin{table}[H]
  \centering
  \begin{tabular}{|l|l|}
    \hline
    \textbf{Command} & \textbf{Result} \\
    \hline
    GET session:user\texttt{1001} & (\texttt{nil}) \\
    GET session:user\texttt{1002} & (\texttt{nil}) \\
    TTL session:user\texttt{1001} & \texttt{-2} (key expired) \\
    TTL session:user\texttt{1002} & \texttt{-2} (key expired) \\
    \hline
  \end{tabular}
  \caption{Session expiration results. See Appendix Figure~\ref{fig:session-expiration} for screenshot.}
\end{table}

A \texttt{TTL} value of -2 indicates that the key has expired and been automatically deleted by Redis.

\section{Visitor Tracking with INCR Operations}

Redis supports atomic increment and decrement operations that work well for counters, analytics, and tracking systems using \texttt{INCR} and \texttt{DECR} commands \parencite{redis_documentation}. These operations are thread-safe and can handle high-concurrency scenarios \parencite{redis_paper}.

\subsection{Basic Visitor Counting}

\begin{minted}[linenos,breaklines]{text}
# Initialize visitor counter
SET visitors:total 0

# Simulate page visits
INCR visitors:total
INCR visitors:total
INCR visitors:total
INCR visitors:total
INCR visitors:total

# Check total visitors
GET visitors:total

# Page-specific visitor tracking
INCR visitors:page:home
INCR visitors:page:home
INCR visitors:page:about
INCR visitors:page:products

# Check page-specific visitors
GET visitors:page:home
GET visitors:page:about
GET visitors:page:products
\end{minted}

\begin{table}[H]
  \centering
  \begin{tabular}{|l|l|}
    \hline
    \textbf{Counter} & \textbf{Value} \\
    \hline
    visitors:total & \texttt{5} \\
    visitors:page:home & \texttt{2} \\
    visitors:page:about & \texttt{1} \\
    visitors:page:products & \texttt{1} \\
    \hline
  \end{tabular}
  \caption{Basic visitor counting results.}
\end{table}

\subsection{Extended Analytics Tracking}

Redis increment operations support various analytics patterns:

\begin{minted}[linenos,breaklines]{text}
# Daily visitor tracking
INCR visitors:daily:2024-01-15
INCR visitors:daily:2024-01-15
INCR visitors:daily:2024-01-15
GET visitors:daily:2024-01-15

# Hourly visitor tracking
INCR visitors:hourly:2024-01-15:10
INCR visitors:hourly:2024-01-15:10
GET visitors:hourly:2024-01-15:10

# User-specific visit tracking
INCR user:1001:visits
GET user:1001:visits

# Increment by specific amount
INCRBY visitors:total 10
GET visitors:total

# Browser and device tracking
INCR browser:chrome
GET browser:chrome
INCR visitors:country:USA
GET visitors:country:USA
INCR visitors:device:desktop
INCR visitors:device:mobile
GET visitors:device:desktop
\end{minted}

\begin{table}[H]
  \centering
  \begin{tabular}{|l|l|}
    \hline
    \textbf{Metric} & \textbf{Value} \\
    \hline
    visitors:daily:\texttt{2024-01-15} & \texttt{3} \\
    visitors:hourly:\texttt{2024-01-15:10} & \texttt{2} \\
    user:\texttt{1001}:visits & \texttt{1} \\
    visitors:total (after INCRBY \texttt{10}) & \texttt{15} \\
    browser:chrome & \texttt{1} \\
    visitors:country:USA & \texttt{1} \\
    visitors:device:desktop & \texttt{2} \\
    \hline
  \end{tabular}
  \caption{Extended analytics tracking results. See Appendix Figure~\ref{fig:visitor-tracking} for screenshot.}
\end{table}

\section{Performance Analysis}

The Redis implementation revealed several operational characteristics in this session management and visitor tracking workload. Redis's in-memory architecture targets specific access pattern requirements \parencite{redis_paper}. The complete source code and implementation can be found at: \url{https://github.com/saileshbro/newsql-comparision/blob/main/task-4/basic-operations.redis}.

\begin{itemize}
    \item \textbf{Memory-Based Access}: All operations executed following Redis's in-memory storage design patterns \parencite{redis_documentation}
    \item \textbf{Atomic Operations}: INCR operations provided thread-safe counting with guaranteed consistency following Redis specifications \parencite{redis_paper}
    \item \textbf{TTL Functionality}: Automatic key expiration operated following Redis's time-to-live implementation \parencite{redis_ttl}
    \item \textbf{Hash Operations}: Complex object storage with field-level access functioned as intended \parencite{redis_documentation}
    \item \textbf{Memory Management}: Automatic cleanup of expired keys operated following Redis's memory management design \parencite{redis_ttl}
\end{itemize}

\section{Use Cases and Applications}

Redis's key-value architecture supports several application patterns \parencite{redis_documentation}:

\begin{itemize}
    \item \textbf{Caching}: Data caching with automatic expiration and invalidation following Redis design \parencite{redis_paper}
    \item \textbf{Session Management}: User session storage with automatic cleanup and TTL support \parencite{redis_ttl}
    \item \textbf{Real-Time Analytics}: Atomic counters for visitor tracking and metrics collection \parencite{redis_documentation}
    \item \textbf{Leaderboards}: Sorted sets for gaming and ranking applications \parencite{redis_paper}
    \item \textbf{Message Queues}: Pub/sub messaging for real-time communication systems \parencite{redis_documentation}
\end{itemize}

The session management and visitor tracking implementation illustrates how Redis handles in-memory data operations with atomic functionality and automatic expiration, showing the database's strength for applications requiring low-latency data processing and caching capabilities \parencite{redis_paper}.